
% Default to the notebook output style

    


% Inherit from the specified cell style.




    
\documentclass[11pt]{article}

    
    
    \usepackage[T1]{fontenc}
    % Nicer default font (+ math font) than Computer Modern for most use cases
    \usepackage{mathpazo}

    % Basic figure setup, for now with no caption control since it's done
    % automatically by Pandoc (which extracts ![](path) syntax from Markdown).
    \usepackage{graphicx}
    % We will generate all images so they have a width \maxwidth. This means
    % that they will get their normal width if they fit onto the page, but
    % are scaled down if they would overflow the margins.
    \makeatletter
    \def\maxwidth{\ifdim\Gin@nat@width>\linewidth\linewidth
    \else\Gin@nat@width\fi}
    \makeatother
    \let\Oldincludegraphics\includegraphics
    % Set max figure width to be 80% of text width, for now hardcoded.
    \renewcommand{\includegraphics}[1]{\Oldincludegraphics[width=.8\maxwidth]{#1}}
    % Ensure that by default, figures have no caption (until we provide a
    % proper Figure object with a Caption API and a way to capture that
    % in the conversion process - todo).
    \usepackage{caption}
    \DeclareCaptionLabelFormat{nolabel}{}
    \captionsetup{labelformat=nolabel}

    \usepackage{adjustbox} % Used to constrain images to a maximum size 
    \usepackage{xcolor} % Allow colors to be defined
    \usepackage{enumerate} % Needed for markdown enumerations to work
    \usepackage{geometry} % Used to adjust the document margins
    \usepackage{amsmath} % Equations
    \usepackage{amssymb} % Equations
    \usepackage{textcomp} % defines textquotesingle
    % Hack from http://tex.stackexchange.com/a/47451/13684:
    \AtBeginDocument{%
        \def\PYZsq{\textquotesingle}% Upright quotes in Pygmentized code
    }
    \usepackage{upquote} % Upright quotes for verbatim code
    \usepackage{eurosym} % defines \euro
    \usepackage[mathletters]{ucs} % Extended unicode (utf-8) support
    \usepackage[utf8x]{inputenc} % Allow utf-8 characters in the tex document
    \usepackage{fancyvrb} % verbatim replacement that allows latex
    \usepackage{grffile} % extends the file name processing of package graphics 
                         % to support a larger range 
    % The hyperref package gives us a pdf with properly built
    % internal navigation ('pdf bookmarks' for the table of contents,
    % internal cross-reference links, web links for URLs, etc.)
    \usepackage{hyperref}
    \usepackage{longtable} % longtable support required by pandoc >1.10
    \usepackage{booktabs}  % table support for pandoc > 1.12.2
    \usepackage[inline]{enumitem} % IRkernel/repr support (it uses the enumerate* environment)
    \usepackage[normalem]{ulem} % ulem is needed to support strikethroughs (\sout)
                                % normalem makes italics be italics, not underlines
    

    
    
    % Colors for the hyperref package
    \definecolor{urlcolor}{rgb}{0,.145,.698}
    \definecolor{linkcolor}{rgb}{.71,0.21,0.01}
    \definecolor{citecolor}{rgb}{.12,.54,.11}

    % ANSI colors
    \definecolor{ansi-black}{HTML}{3E424D}
    \definecolor{ansi-black-intense}{HTML}{282C36}
    \definecolor{ansi-red}{HTML}{E75C58}
    \definecolor{ansi-red-intense}{HTML}{B22B31}
    \definecolor{ansi-green}{HTML}{00A250}
    \definecolor{ansi-green-intense}{HTML}{007427}
    \definecolor{ansi-yellow}{HTML}{DDB62B}
    \definecolor{ansi-yellow-intense}{HTML}{B27D12}
    \definecolor{ansi-blue}{HTML}{208FFB}
    \definecolor{ansi-blue-intense}{HTML}{0065CA}
    \definecolor{ansi-magenta}{HTML}{D160C4}
    \definecolor{ansi-magenta-intense}{HTML}{A03196}
    \definecolor{ansi-cyan}{HTML}{60C6C8}
    \definecolor{ansi-cyan-intense}{HTML}{258F8F}
    \definecolor{ansi-white}{HTML}{C5C1B4}
    \definecolor{ansi-white-intense}{HTML}{A1A6B2}

    % commands and environments needed by pandoc snippets
    % extracted from the output of `pandoc -s`
    \providecommand{\tightlist}{%
      \setlength{\itemsep}{0pt}\setlength{\parskip}{0pt}}
    \DefineVerbatimEnvironment{Highlighting}{Verbatim}{commandchars=\\\{\}}
    % Add ',fontsize=\small' for more characters per line
    \newenvironment{Shaded}{}{}
    \newcommand{\KeywordTok}[1]{\textcolor[rgb]{0.00,0.44,0.13}{\textbf{{#1}}}}
    \newcommand{\DataTypeTok}[1]{\textcolor[rgb]{0.56,0.13,0.00}{{#1}}}
    \newcommand{\DecValTok}[1]{\textcolor[rgb]{0.25,0.63,0.44}{{#1}}}
    \newcommand{\BaseNTok}[1]{\textcolor[rgb]{0.25,0.63,0.44}{{#1}}}
    \newcommand{\FloatTok}[1]{\textcolor[rgb]{0.25,0.63,0.44}{{#1}}}
    \newcommand{\CharTok}[1]{\textcolor[rgb]{0.25,0.44,0.63}{{#1}}}
    \newcommand{\StringTok}[1]{\textcolor[rgb]{0.25,0.44,0.63}{{#1}}}
    \newcommand{\CommentTok}[1]{\textcolor[rgb]{0.38,0.63,0.69}{\textit{{#1}}}}
    \newcommand{\OtherTok}[1]{\textcolor[rgb]{0.00,0.44,0.13}{{#1}}}
    \newcommand{\AlertTok}[1]{\textcolor[rgb]{1.00,0.00,0.00}{\textbf{{#1}}}}
    \newcommand{\FunctionTok}[1]{\textcolor[rgb]{0.02,0.16,0.49}{{#1}}}
    \newcommand{\RegionMarkerTok}[1]{{#1}}
    \newcommand{\ErrorTok}[1]{\textcolor[rgb]{1.00,0.00,0.00}{\textbf{{#1}}}}
    \newcommand{\NormalTok}[1]{{#1}}
    
    % Additional commands for more recent versions of Pandoc
    \newcommand{\ConstantTok}[1]{\textcolor[rgb]{0.53,0.00,0.00}{{#1}}}
    \newcommand{\SpecialCharTok}[1]{\textcolor[rgb]{0.25,0.44,0.63}{{#1}}}
    \newcommand{\VerbatimStringTok}[1]{\textcolor[rgb]{0.25,0.44,0.63}{{#1}}}
    \newcommand{\SpecialStringTok}[1]{\textcolor[rgb]{0.73,0.40,0.53}{{#1}}}
    \newcommand{\ImportTok}[1]{{#1}}
    \newcommand{\DocumentationTok}[1]{\textcolor[rgb]{0.73,0.13,0.13}{\textit{{#1}}}}
    \newcommand{\AnnotationTok}[1]{\textcolor[rgb]{0.38,0.63,0.69}{\textbf{\textit{{#1}}}}}
    \newcommand{\CommentVarTok}[1]{\textcolor[rgb]{0.38,0.63,0.69}{\textbf{\textit{{#1}}}}}
    \newcommand{\VariableTok}[1]{\textcolor[rgb]{0.10,0.09,0.49}{{#1}}}
    \newcommand{\ControlFlowTok}[1]{\textcolor[rgb]{0.00,0.44,0.13}{\textbf{{#1}}}}
    \newcommand{\OperatorTok}[1]{\textcolor[rgb]{0.40,0.40,0.40}{{#1}}}
    \newcommand{\BuiltInTok}[1]{{#1}}
    \newcommand{\ExtensionTok}[1]{{#1}}
    \newcommand{\PreprocessorTok}[1]{\textcolor[rgb]{0.74,0.48,0.00}{{#1}}}
    \newcommand{\AttributeTok}[1]{\textcolor[rgb]{0.49,0.56,0.16}{{#1}}}
    \newcommand{\InformationTok}[1]{\textcolor[rgb]{0.38,0.63,0.69}{\textbf{\textit{{#1}}}}}
    \newcommand{\WarningTok}[1]{\textcolor[rgb]{0.38,0.63,0.69}{\textbf{\textit{{#1}}}}}
    
    
    % Define a nice break command that doesn't care if a line doesn't already
    % exist.
    \def\br{\hspace*{\fill} \\* }
    % Math Jax compatability definitions
    \def\gt{>}
    \def\lt{<}
    % Document parameters
    \title{John Snow Ghost Map}
    
    
    

    % Pygments definitions
    
\makeatletter
\def\PY@reset{\let\PY@it=\relax \let\PY@bf=\relax%
    \let\PY@ul=\relax \let\PY@tc=\relax%
    \let\PY@bc=\relax \let\PY@ff=\relax}
\def\PY@tok#1{\csname PY@tok@#1\endcsname}
\def\PY@toks#1+{\ifx\relax#1\empty\else%
    \PY@tok{#1}\expandafter\PY@toks\fi}
\def\PY@do#1{\PY@bc{\PY@tc{\PY@ul{%
    \PY@it{\PY@bf{\PY@ff{#1}}}}}}}
\def\PY#1#2{\PY@reset\PY@toks#1+\relax+\PY@do{#2}}

\expandafter\def\csname PY@tok@w\endcsname{\def\PY@tc##1{\textcolor[rgb]{0.73,0.73,0.73}{##1}}}
\expandafter\def\csname PY@tok@c\endcsname{\let\PY@it=\textit\def\PY@tc##1{\textcolor[rgb]{0.25,0.50,0.50}{##1}}}
\expandafter\def\csname PY@tok@cp\endcsname{\def\PY@tc##1{\textcolor[rgb]{0.74,0.48,0.00}{##1}}}
\expandafter\def\csname PY@tok@k\endcsname{\let\PY@bf=\textbf\def\PY@tc##1{\textcolor[rgb]{0.00,0.50,0.00}{##1}}}
\expandafter\def\csname PY@tok@kp\endcsname{\def\PY@tc##1{\textcolor[rgb]{0.00,0.50,0.00}{##1}}}
\expandafter\def\csname PY@tok@kt\endcsname{\def\PY@tc##1{\textcolor[rgb]{0.69,0.00,0.25}{##1}}}
\expandafter\def\csname PY@tok@o\endcsname{\def\PY@tc##1{\textcolor[rgb]{0.40,0.40,0.40}{##1}}}
\expandafter\def\csname PY@tok@ow\endcsname{\let\PY@bf=\textbf\def\PY@tc##1{\textcolor[rgb]{0.67,0.13,1.00}{##1}}}
\expandafter\def\csname PY@tok@nb\endcsname{\def\PY@tc##1{\textcolor[rgb]{0.00,0.50,0.00}{##1}}}
\expandafter\def\csname PY@tok@nf\endcsname{\def\PY@tc##1{\textcolor[rgb]{0.00,0.00,1.00}{##1}}}
\expandafter\def\csname PY@tok@nc\endcsname{\let\PY@bf=\textbf\def\PY@tc##1{\textcolor[rgb]{0.00,0.00,1.00}{##1}}}
\expandafter\def\csname PY@tok@nn\endcsname{\let\PY@bf=\textbf\def\PY@tc##1{\textcolor[rgb]{0.00,0.00,1.00}{##1}}}
\expandafter\def\csname PY@tok@ne\endcsname{\let\PY@bf=\textbf\def\PY@tc##1{\textcolor[rgb]{0.82,0.25,0.23}{##1}}}
\expandafter\def\csname PY@tok@nv\endcsname{\def\PY@tc##1{\textcolor[rgb]{0.10,0.09,0.49}{##1}}}
\expandafter\def\csname PY@tok@no\endcsname{\def\PY@tc##1{\textcolor[rgb]{0.53,0.00,0.00}{##1}}}
\expandafter\def\csname PY@tok@nl\endcsname{\def\PY@tc##1{\textcolor[rgb]{0.63,0.63,0.00}{##1}}}
\expandafter\def\csname PY@tok@ni\endcsname{\let\PY@bf=\textbf\def\PY@tc##1{\textcolor[rgb]{0.60,0.60,0.60}{##1}}}
\expandafter\def\csname PY@tok@na\endcsname{\def\PY@tc##1{\textcolor[rgb]{0.49,0.56,0.16}{##1}}}
\expandafter\def\csname PY@tok@nt\endcsname{\let\PY@bf=\textbf\def\PY@tc##1{\textcolor[rgb]{0.00,0.50,0.00}{##1}}}
\expandafter\def\csname PY@tok@nd\endcsname{\def\PY@tc##1{\textcolor[rgb]{0.67,0.13,1.00}{##1}}}
\expandafter\def\csname PY@tok@s\endcsname{\def\PY@tc##1{\textcolor[rgb]{0.73,0.13,0.13}{##1}}}
\expandafter\def\csname PY@tok@sd\endcsname{\let\PY@it=\textit\def\PY@tc##1{\textcolor[rgb]{0.73,0.13,0.13}{##1}}}
\expandafter\def\csname PY@tok@si\endcsname{\let\PY@bf=\textbf\def\PY@tc##1{\textcolor[rgb]{0.73,0.40,0.53}{##1}}}
\expandafter\def\csname PY@tok@se\endcsname{\let\PY@bf=\textbf\def\PY@tc##1{\textcolor[rgb]{0.73,0.40,0.13}{##1}}}
\expandafter\def\csname PY@tok@sr\endcsname{\def\PY@tc##1{\textcolor[rgb]{0.73,0.40,0.53}{##1}}}
\expandafter\def\csname PY@tok@ss\endcsname{\def\PY@tc##1{\textcolor[rgb]{0.10,0.09,0.49}{##1}}}
\expandafter\def\csname PY@tok@sx\endcsname{\def\PY@tc##1{\textcolor[rgb]{0.00,0.50,0.00}{##1}}}
\expandafter\def\csname PY@tok@m\endcsname{\def\PY@tc##1{\textcolor[rgb]{0.40,0.40,0.40}{##1}}}
\expandafter\def\csname PY@tok@gh\endcsname{\let\PY@bf=\textbf\def\PY@tc##1{\textcolor[rgb]{0.00,0.00,0.50}{##1}}}
\expandafter\def\csname PY@tok@gu\endcsname{\let\PY@bf=\textbf\def\PY@tc##1{\textcolor[rgb]{0.50,0.00,0.50}{##1}}}
\expandafter\def\csname PY@tok@gd\endcsname{\def\PY@tc##1{\textcolor[rgb]{0.63,0.00,0.00}{##1}}}
\expandafter\def\csname PY@tok@gi\endcsname{\def\PY@tc##1{\textcolor[rgb]{0.00,0.63,0.00}{##1}}}
\expandafter\def\csname PY@tok@gr\endcsname{\def\PY@tc##1{\textcolor[rgb]{1.00,0.00,0.00}{##1}}}
\expandafter\def\csname PY@tok@ge\endcsname{\let\PY@it=\textit}
\expandafter\def\csname PY@tok@gs\endcsname{\let\PY@bf=\textbf}
\expandafter\def\csname PY@tok@gp\endcsname{\let\PY@bf=\textbf\def\PY@tc##1{\textcolor[rgb]{0.00,0.00,0.50}{##1}}}
\expandafter\def\csname PY@tok@go\endcsname{\def\PY@tc##1{\textcolor[rgb]{0.53,0.53,0.53}{##1}}}
\expandafter\def\csname PY@tok@gt\endcsname{\def\PY@tc##1{\textcolor[rgb]{0.00,0.27,0.87}{##1}}}
\expandafter\def\csname PY@tok@err\endcsname{\def\PY@bc##1{\setlength{\fboxsep}{0pt}\fcolorbox[rgb]{1.00,0.00,0.00}{1,1,1}{\strut ##1}}}
\expandafter\def\csname PY@tok@kc\endcsname{\let\PY@bf=\textbf\def\PY@tc##1{\textcolor[rgb]{0.00,0.50,0.00}{##1}}}
\expandafter\def\csname PY@tok@kd\endcsname{\let\PY@bf=\textbf\def\PY@tc##1{\textcolor[rgb]{0.00,0.50,0.00}{##1}}}
\expandafter\def\csname PY@tok@kn\endcsname{\let\PY@bf=\textbf\def\PY@tc##1{\textcolor[rgb]{0.00,0.50,0.00}{##1}}}
\expandafter\def\csname PY@tok@kr\endcsname{\let\PY@bf=\textbf\def\PY@tc##1{\textcolor[rgb]{0.00,0.50,0.00}{##1}}}
\expandafter\def\csname PY@tok@bp\endcsname{\def\PY@tc##1{\textcolor[rgb]{0.00,0.50,0.00}{##1}}}
\expandafter\def\csname PY@tok@fm\endcsname{\def\PY@tc##1{\textcolor[rgb]{0.00,0.00,1.00}{##1}}}
\expandafter\def\csname PY@tok@vc\endcsname{\def\PY@tc##1{\textcolor[rgb]{0.10,0.09,0.49}{##1}}}
\expandafter\def\csname PY@tok@vg\endcsname{\def\PY@tc##1{\textcolor[rgb]{0.10,0.09,0.49}{##1}}}
\expandafter\def\csname PY@tok@vi\endcsname{\def\PY@tc##1{\textcolor[rgb]{0.10,0.09,0.49}{##1}}}
\expandafter\def\csname PY@tok@vm\endcsname{\def\PY@tc##1{\textcolor[rgb]{0.10,0.09,0.49}{##1}}}
\expandafter\def\csname PY@tok@sa\endcsname{\def\PY@tc##1{\textcolor[rgb]{0.73,0.13,0.13}{##1}}}
\expandafter\def\csname PY@tok@sb\endcsname{\def\PY@tc##1{\textcolor[rgb]{0.73,0.13,0.13}{##1}}}
\expandafter\def\csname PY@tok@sc\endcsname{\def\PY@tc##1{\textcolor[rgb]{0.73,0.13,0.13}{##1}}}
\expandafter\def\csname PY@tok@dl\endcsname{\def\PY@tc##1{\textcolor[rgb]{0.73,0.13,0.13}{##1}}}
\expandafter\def\csname PY@tok@s2\endcsname{\def\PY@tc##1{\textcolor[rgb]{0.73,0.13,0.13}{##1}}}
\expandafter\def\csname PY@tok@sh\endcsname{\def\PY@tc##1{\textcolor[rgb]{0.73,0.13,0.13}{##1}}}
\expandafter\def\csname PY@tok@s1\endcsname{\def\PY@tc##1{\textcolor[rgb]{0.73,0.13,0.13}{##1}}}
\expandafter\def\csname PY@tok@mb\endcsname{\def\PY@tc##1{\textcolor[rgb]{0.40,0.40,0.40}{##1}}}
\expandafter\def\csname PY@tok@mf\endcsname{\def\PY@tc##1{\textcolor[rgb]{0.40,0.40,0.40}{##1}}}
\expandafter\def\csname PY@tok@mh\endcsname{\def\PY@tc##1{\textcolor[rgb]{0.40,0.40,0.40}{##1}}}
\expandafter\def\csname PY@tok@mi\endcsname{\def\PY@tc##1{\textcolor[rgb]{0.40,0.40,0.40}{##1}}}
\expandafter\def\csname PY@tok@il\endcsname{\def\PY@tc##1{\textcolor[rgb]{0.40,0.40,0.40}{##1}}}
\expandafter\def\csname PY@tok@mo\endcsname{\def\PY@tc##1{\textcolor[rgb]{0.40,0.40,0.40}{##1}}}
\expandafter\def\csname PY@tok@ch\endcsname{\let\PY@it=\textit\def\PY@tc##1{\textcolor[rgb]{0.25,0.50,0.50}{##1}}}
\expandafter\def\csname PY@tok@cm\endcsname{\let\PY@it=\textit\def\PY@tc##1{\textcolor[rgb]{0.25,0.50,0.50}{##1}}}
\expandafter\def\csname PY@tok@cpf\endcsname{\let\PY@it=\textit\def\PY@tc##1{\textcolor[rgb]{0.25,0.50,0.50}{##1}}}
\expandafter\def\csname PY@tok@c1\endcsname{\let\PY@it=\textit\def\PY@tc##1{\textcolor[rgb]{0.25,0.50,0.50}{##1}}}
\expandafter\def\csname PY@tok@cs\endcsname{\let\PY@it=\textit\def\PY@tc##1{\textcolor[rgb]{0.25,0.50,0.50}{##1}}}

\def\PYZbs{\char`\\}
\def\PYZus{\char`\_}
\def\PYZob{\char`\{}
\def\PYZcb{\char`\}}
\def\PYZca{\char`\^}
\def\PYZam{\char`\&}
\def\PYZlt{\char`\<}
\def\PYZgt{\char`\>}
\def\PYZsh{\char`\#}
\def\PYZpc{\char`\%}
\def\PYZdl{\char`\$}
\def\PYZhy{\char`\-}
\def\PYZsq{\char`\'}
\def\PYZdq{\char`\"}
\def\PYZti{\char`\~}
% for compatibility with earlier versions
\def\PYZat{@}
\def\PYZlb{[}
\def\PYZrb{]}
\makeatother


    % Exact colors from NB
    \definecolor{incolor}{rgb}{0.0, 0.0, 0.5}
    \definecolor{outcolor}{rgb}{0.545, 0.0, 0.0}



    
    % Prevent overflowing lines due to hard-to-break entities
    \sloppy 
    % Setup hyperref package
    \hypersetup{
      breaklinks=true,  % so long urls are correctly broken across lines
      colorlinks=true,
      urlcolor=urlcolor,
      linkcolor=linkcolor,
      citecolor=citecolor,
      }
    % Slightly bigger margins than the latex defaults
    
    \geometry{verbose,tmargin=1in,bmargin=1in,lmargin=1in,rmargin=1in}
    
    

    \begin{document}
    
    
    \maketitle
    
    

    
    \subsection{1. Dr. John Snow}\label{dr.-john-snow}

Dr. John Snow (1813-1858) was a famous British physician and is widely
recognized as a legendary figure in the history of public health and a
leading pioneer in the development of anesthesia. Some even say one of
the greatest physicians of all time.

As a leading advocate of both anesthesia and hygienic practices in
medicine, he not only experimented with ether and chloroform but also
designed a mask and method how to administer it. He personally
administered chloroform to Queen Victoria during the births of her
eighth and ninth children, in 1853 and 1857, which assured a growing
public acceptance of the use of anesthetics during childbirth.

But, as we will show later, not all his life was just a success. John
Snow is now also recognized as one of the founders of modern
epidemiology (some also consider him as the founder of data
visualization, spatial analysis, data science in general, and many other
related fields) for his scientific and pretty modern data approach in
identifying the source of a cholera outbreak in Soho, London in 1854,
but it wasn't always like this. In fact, for a long time, he was simply
ignored by the scientific community and currently is very often
mythified.

In this notebook, we're not only going to rediscover his "data story",
but reanalyze the data that he collected in 1854 and recreate his famous
map (also called The Ghost Map).

    \begin{Verbatim}[commandchars=\\\{\}]
{\color{incolor}In [{\color{incolor}36}]:} \PY{c+c1}{\PYZsh{} Loading in the pandas module}
         \PY{c+c1}{\PYZsh{} ... YOUR CODE FOR TASK 1 ...}
         \PY{k+kn}{import} \PY{n+nn}{pandas} \PY{k}{as} \PY{n+nn}{pd}
         \PY{k+kn}{import} \PY{n+nn}{matplotlib}\PY{n+nn}{.}\PY{n+nn}{pyplot} \PY{k}{as} \PY{n+nn}{plt}
         
         \PY{c+c1}{\PYZsh{} Reading in the data}
         \PY{n}{deaths} \PY{o}{=} \PY{n}{pd}\PY{o}{.}\PY{n}{read\PYZus{}csv}\PY{p}{(}\PY{l+s+s1}{\PYZsq{}}\PY{l+s+s1}{datasets/deaths.csv}\PY{l+s+s1}{\PYZsq{}}\PY{p}{)}
         
         \PY{c+c1}{\PYZsh{} Print out the shape of the dataset}
         \PY{c+c1}{\PYZsh{} ... YOUR CODE FOR TASK 1 ...}
         \PY{n}{deaths}\PY{o}{.}\PY{n}{shape}
         \PY{c+c1}{\PYZsh{} Printing out the first 5 rows}
         \PY{c+c1}{\PYZsh{} ... YOUR CODE FOR TASK 1 ...}
         \PY{n}{deaths}\PY{o}{.}\PY{n}{head}\PY{p}{(}\PY{p}{)}
\end{Verbatim}


\begin{Verbatim}[commandchars=\\\{\}]
{\color{outcolor}Out[{\color{outcolor}36}]:}    Death  X coordinate  Y coordinate
         0      1     51.513418     -0.137930
         1      1     51.513418     -0.137930
         2      1     51.513418     -0.137930
         3      1     51.513361     -0.137883
         4      1     51.513361     -0.137883
\end{Verbatim}
            
    \subsection{2. Cholera attacks!}\label{cholera-attacks}

Prior to John Snow's discovery cholera was a regular visitor to London's
overcrowded and unsanitary streets. During the time of the third cholera
outbreak, it was one of the most studied subjects (between years
1839-1856 over 700 studies and essays were published in London alone)
and nearly all of the authors believed the outbreaks were due to miasma
or "bad air".

It was John Snow's pioneering work with anesthesia and gases that made
him doubt the miasma model of the disease. Originally he formulated and
published his theory that cholera is spread by water or food in an essay
On the Mode of Communication of Cholera (before the outbreak in 1849).
The essay received negative reviews in the Lancet and the London Medical
Gazette.

We know now that he was right, but Dr. Snow's dilemma was how to prove
it? His first step to getting there was checking the data. Our dataset
has 489 rows of data in 3 columns but to work with dataset more easily
we will first make few changes.

    \begin{Verbatim}[commandchars=\\\{\}]
{\color{incolor}In [{\color{incolor}38}]:} \PY{c+c1}{\PYZsh{} Summarizing the content of deaths}
         \PY{c+c1}{\PYZsh{} ... YOUR CODE FOR TASK 2 ...}
         \PY{n}{deaths}\PY{o}{.}\PY{n}{info}\PY{p}{(}\PY{p}{)}
         \PY{c+c1}{\PYZsh{} Define the new names of your columns}
         \PY{n}{newcols} \PY{o}{=} \PY{p}{\PYZob{}}\PY{l+s+s1}{\PYZsq{}}\PY{l+s+s1}{Death}\PY{l+s+s1}{\PYZsq{}}\PY{p}{:} \PY{l+s+s1}{\PYZsq{}}\PY{l+s+s1}{death\PYZus{}count}\PY{l+s+s1}{\PYZsq{}}\PY{p}{,}\PY{l+s+s1}{\PYZsq{}}\PY{l+s+s1}{X coordinate}\PY{l+s+s1}{\PYZsq{}}\PY{p}{:} \PY{l+s+s1}{\PYZsq{}}\PY{l+s+s1}{x\PYZus{}latitude}\PY{l+s+s1}{\PYZsq{}}\PY{p}{,} \PY{l+s+s1}{\PYZsq{}}\PY{l+s+s1}{Y coordinate}\PY{l+s+s1}{\PYZsq{}}\PY{p}{:} \PY{l+s+s1}{\PYZsq{}}\PY{l+s+s1}{y\PYZus{}longitude}\PY{l+s+s1}{\PYZsq{}} \PY{p}{\PYZcb{}}
         
         \PY{c+c1}{\PYZsh{} Rename your columns}
         \PY{c+c1}{\PYZsh{} ... YOUR CODE FOR TASK 2 ...}
         \PY{n}{deaths}\PY{o}{.}\PY{n}{rename}\PY{p}{(}\PY{n}{columns} \PY{o}{=} \PY{n}{newcols}\PY{p}{,} \PY{n}{inplace} \PY{o}{=}\PY{k+kc}{True}\PY{p}{)}
         \PY{c+c1}{\PYZsh{} Describe the dataset }
         \PY{c+c1}{\PYZsh{} ... YOUR CODE FOR TASK 2 ...}
         \PY{n}{deaths}\PY{o}{.}\PY{n}{describe}\PY{p}{(}\PY{p}{)}
\end{Verbatim}


    \begin{Verbatim}[commandchars=\\\{\}]
<class 'pandas.core.frame.DataFrame'>
RangeIndex: 489 entries, 0 to 488
Data columns (total 3 columns):
Death           489 non-null int64
X coordinate    489 non-null float64
Y coordinate    489 non-null float64
dtypes: float64(2), int64(1)
memory usage: 11.5 KB

    \end{Verbatim}

\begin{Verbatim}[commandchars=\\\{\}]
{\color{outcolor}Out[{\color{outcolor}38}]:}        death\_count  x\_latitude  y\_longitude
         count        489.0  489.000000   489.000000
         mean           1.0   51.513398    -0.136403
         std            0.0    0.000705     0.001503
         min            1.0   51.511856    -0.140074
         25\%            1.0   51.512964    -0.137562
         50\%            1.0   51.513359    -0.136226
         75\%            1.0   51.513875    -0.135344
         max            1.0   51.515834    -0.132933
\end{Verbatim}
            
    \subsection{3. You know nothing, John
Snow!}\label{you-know-nothing-john-snow}

It was somehow unthinkable that one man could debunk the miasma theory
and prove that all the others got it wrong, so his work was mostly
ignored. His medical colleagues simply said: "You know nothing, John
Snow!"

As already mentioned John Snow's first attempt to debunk the "miasma"
theory ended with negative reviews. However, a reviewer made a helpful
suggestion in terms of what evidence would be compelling: the crucial
natural experiment would be to find people living side by side with
lifestyles similar in all respects except for the water source. The
cholera outbreak in Soho, London in 1854 gave Snow the opportunity not
only to save lives this time but also to further test and improve his
theory. But what about the final proof that he was right?

We now know how John Snow did it, so let's get the data right first.

    \begin{Verbatim}[commandchars=\\\{\}]
{\color{incolor}In [{\color{incolor}40}]:} \PY{c+c1}{\PYZsh{} Create `locations` by subsetting only Latitude and Longitude from the dataset }
         \PY{n}{locations} \PY{o}{=} \PY{n}{deaths}\PY{p}{[}\PY{p}{[}\PY{l+s+s1}{\PYZsq{}}\PY{l+s+s1}{x\PYZus{}latitude}\PY{l+s+s1}{\PYZsq{}}\PY{p}{,}\PY{l+s+s1}{\PYZsq{}}\PY{l+s+s1}{y\PYZus{}longitude}\PY{l+s+s1}{\PYZsq{}}\PY{p}{]}\PY{p}{]}
         
         \PY{c+c1}{\PYZsh{} Create `deaths\PYZus{}list` by transforming the DataFrame to list of lists }
         \PY{n}{deaths\PYZus{}list} \PY{o}{=} \PY{n}{locations}\PY{o}{.}\PY{n}{values}\PY{o}{.}\PY{n}{tolist}\PY{p}{(}\PY{p}{)}
         
         \PY{c+c1}{\PYZsh{} Check the length of the list}
         \PY{c+c1}{\PYZsh{} ... YOUR CODE FOR TASK 3 ...}
         \PY{n+nb}{len}\PY{p}{(}\PY{n}{deaths\PYZus{}list}\PY{p}{)}
\end{Verbatim}


\begin{Verbatim}[commandchars=\\\{\}]
{\color{outcolor}Out[{\color{outcolor}40}]:} 489
\end{Verbatim}
            
    \subsection{4. The Ghost Map}\label{the-ghost-map}

His original map, unfortunately, is not available (it might never even
existed). We can see the famous one that he drew about a year later in
1855, though, and it is displayed in this cell. Because the map depicts
and visualizes the deaths sometimes it is called also The Ghost Map.

We now know how John Snow did it and have the data too, so let's
recreate his map using modern techniques.

    \begin{Verbatim}[commandchars=\\\{\}]
{\color{incolor}In [{\color{incolor}42}]:} \PY{c+c1}{\PYZsh{} Plot the data on map (map location is provided) using folium and for loop for plotting all the points}
         \PY{k+kn}{import} \PY{n+nn}{folium}
         
         \PY{n+nb}{map} \PY{o}{=} \PY{n}{folium}\PY{o}{.}\PY{n}{Map}\PY{p}{(}\PY{n}{location}\PY{o}{=}\PY{p}{[}\PY{l+m+mf}{51.5132119}\PY{p}{,}\PY{o}{\PYZhy{}}\PY{l+m+mf}{0.13666}\PY{p}{]}\PY{p}{,} \PY{n}{tiles}\PY{o}{=}\PY{l+s+s1}{\PYZsq{}}\PY{l+s+s1}{Stamen Toner}\PY{l+s+s1}{\PYZsq{}}\PY{p}{,} \PY{n}{zoom\PYZus{}start}\PY{o}{=}\PY{l+m+mi}{17}\PY{p}{)}
         \PY{k}{for} \PY{n}{point} \PY{o+ow}{in} \PY{n+nb}{range}\PY{p}{(}\PY{l+m+mi}{0}\PY{p}{,} \PY{n+nb}{len}\PY{p}{(}\PY{n}{deaths\PYZus{}list}\PY{p}{)}\PY{p}{)}\PY{p}{:}
             \PY{n}{folium}\PY{o}{.}\PY{n}{CircleMarker}\PY{p}{(}\PY{n}{deaths\PYZus{}list}\PY{p}{[}\PY{n}{point}\PY{p}{]}\PY{p}{,} \PY{n}{radius}\PY{o}{=}\PY{l+m+mi}{8}\PY{p}{,} \PY{n}{color}\PY{o}{=}\PY{l+s+s1}{\PYZsq{}}\PY{l+s+s1}{red}\PY{l+s+s1}{\PYZsq{}}\PY{p}{,} \PY{n}{fill}\PY{o}{=}\PY{k+kc}{True}\PY{p}{,} \PY{n}{fill\PYZus{}color}\PY{o}{=}\PY{l+s+s1}{\PYZsq{}}\PY{l+s+s1}{red}\PY{l+s+s1}{\PYZsq{}}\PY{p}{,} \PY{n}{opacity} \PY{o}{=} \PY{l+m+mf}{0.4}\PY{p}{)}\PY{o}{.}\PY{n}{add\PYZus{}to}\PY{p}{(}\PY{n+nb}{map}\PY{p}{)}
         \PY{n+nb}{map}
\end{Verbatim}


\begin{Verbatim}[commandchars=\\\{\}]
{\color{outcolor}Out[{\color{outcolor}42}]:} <folium.folium.Map at 0x7f7578f40ef0>
\end{Verbatim}
            
    \subsection{5. It's the pump!}\label{its-the-pump}

After marking the deaths on the map, what John Snow saw was not a random
pattern (we saw this on our recreation of The Ghost Map too). The
majority of the deaths were concentrated at the corner of Broad Street
(now Broadwick Street) and Cambridge Street (now Lexington Street). A
cluster of deaths around the junction of these streets was the epicenter
of the outbreak, but what was there? Yes, a water pump.

John Snow at the time already had a developed theory that cholera
spreads through water, so to test this he marked on the map also the
locations of the water pumps nearby. And here it was, the whole picture.

By combining the location of deaths related to cholera with locations of
the water pumps, Snow was able to show that the majority were clustered
around one particular public water pump in Broad Street, Soho. Finally,
he had the proof that he needed.

We will now do the same and add the locations of the pumps to our
recreation of The Ghost Map.

    \begin{Verbatim}[commandchars=\\\{\}]
{\color{incolor}In [{\color{incolor}44}]:} \PY{c+c1}{\PYZsh{} Import the data}
         \PY{n}{pumps} \PY{o}{=} \PY{n}{pd}\PY{o}{.}\PY{n}{read\PYZus{}csv}\PY{p}{(}\PY{l+s+s1}{\PYZsq{}}\PY{l+s+s1}{datasets/pumps.csv}\PY{l+s+s1}{\PYZsq{}}\PY{p}{)}
         
         \PY{c+c1}{\PYZsh{} Subset the DataFrame and select just [\PYZsq{}X coordinate\PYZsq{}, \PYZsq{}Y coordinate\PYZsq{}] columns}
         \PY{n}{locations\PYZus{}pumps} \PY{o}{=} \PY{n}{pumps}\PY{p}{[}\PY{p}{[}\PY{l+s+s1}{\PYZsq{}}\PY{l+s+s1}{X coordinate}\PY{l+s+s1}{\PYZsq{}}\PY{p}{,} \PY{l+s+s1}{\PYZsq{}}\PY{l+s+s1}{Y coordinate}\PY{l+s+s1}{\PYZsq{}}\PY{p}{]}\PY{p}{]}
         \PY{c+c1}{\PYZsh{} Transform the DataFrame to list of lists in form of [\PYZsq{}X coordinate\PYZsq{}, \PYZsq{}Y coordinate\PYZsq{}] pairs}
         \PY{n}{pumps\PYZus{}list} \PY{o}{=} \PY{n}{locations\PYZus{}pumps}\PY{o}{.}\PY{n}{values}\PY{o}{.}\PY{n}{tolist}\PY{p}{(}\PY{p}{)}
         
         \PY{c+c1}{\PYZsh{} Create a for loop and plot the data using folium (use previous map + add another layer)}
         \PY{n}{map1} \PY{o}{=} \PY{n+nb}{map}
         \PY{k}{for} \PY{n}{point} \PY{o+ow}{in} \PY{n+nb}{range}\PY{p}{(}\PY{l+m+mi}{0}\PY{p}{,} \PY{n+nb}{len}\PY{p}{(}\PY{n}{pumps\PYZus{}list}\PY{p}{)}\PY{p}{)}\PY{p}{:}
             \PY{n}{folium}\PY{o}{.}\PY{n}{Marker}\PY{p}{(}\PY{n}{pumps\PYZus{}list}\PY{p}{[}\PY{n}{point}\PY{p}{]}\PY{p}{,} \PY{n}{popup}\PY{o}{=}\PY{n}{pumps}\PY{p}{[}\PY{l+s+s1}{\PYZsq{}}\PY{l+s+s1}{Pump Name}\PY{l+s+s1}{\PYZsq{}}\PY{p}{]}\PY{p}{[}\PY{n}{point}\PY{p}{]}\PY{p}{)}\PY{o}{.}\PY{n}{add\PYZus{}to}\PY{p}{(}\PY{n}{map1}\PY{p}{)}
         \PY{n}{map1}
\end{Verbatim}


\begin{Verbatim}[commandchars=\\\{\}]
{\color{outcolor}Out[{\color{outcolor}44}]:} <folium.folium.Map at 0x7f7578f40ef0>
\end{Verbatim}
            
    \subsection{6. You know nothing, John Snow!
(again)}\label{you-know-nothing-john-snow-again}

So, John Snow finally had his proof that there was a connection between
deaths as a consequence of the cholera outbreak and the public water
pump that was probably contaminated. But he didn't just stop there and
investigated further.

He was looking for anomalies now (we would now say "outliers in data")
and found two in fact where there were no deaths. First was brewery
right on the Broad Street, so he went there and learned that they drank
mostly beer (in other words not the water from the local pump, which
confirms his theory that the pump is the source of the outbreak). The
second building without any deaths was workhouse near Poland street
where he learned that their source of water was not the pump on the
Broad Street (confirmation again). The locations of both buildings are
visualized also on the map on the left.

He was now sure, and although officials did not trust him nor his theory
they removed the handle to the pump next day, 8th of September 1854.
John Snow later collected and published in his famous book also all the
data about deaths in chronological order, before and after the peak of
the outbreak and we will now analyze and compare the effect when the
handle was removed.

    \begin{Verbatim}[commandchars=\\\{\}]
{\color{incolor}In [{\color{incolor}63}]:} \PY{c+c1}{\PYZsh{}Importing packages}
         \PY{k+kn}{from} \PY{n+nn}{datetime} \PY{k}{import} \PY{n}{date}
         \PY{k+kn}{import} \PY{n+nn}{calendar}
         
         \PY{c+c1}{\PYZsh{} Import the data the right way}
         \PY{n}{dates} \PY{o}{=} \PY{n}{pd}\PY{o}{.}\PY{n}{read\PYZus{}csv}\PY{p}{(}\PY{l+s+s1}{\PYZsq{}}\PY{l+s+s1}{datasets/dates.csv}\PY{l+s+s1}{\PYZsq{}}\PY{p}{,} \PY{n}{parse\PYZus{}dates}\PY{o}{=}\PY{p}{[}\PY{l+s+s1}{\PYZsq{}}\PY{l+s+s1}{date}\PY{l+s+s1}{\PYZsq{}}\PY{p}{]}\PY{p}{)}
         \PY{c+c1}{\PYZsh{} Set the Date when handle was removed (8th of September 1854)}
         \PY{n}{handle\PYZus{}removed} \PY{o}{=} \PY{n}{pd}\PY{o}{.}\PY{n}{to\PYZus{}datetime}\PY{p}{(}\PY{l+s+s1}{\PYZsq{}}\PY{l+s+s1}{1854/9/8}\PY{l+s+s1}{\PYZsq{}}\PY{p}{)}
         
         \PY{c+c1}{\PYZsh{} Create new column `day\PYZus{}name` in `dates` DataFrame with names of the day }
         \PY{n}{dates}\PY{p}{[}\PY{l+s+s1}{\PYZsq{}}\PY{l+s+s1}{day\PYZus{}name}\PY{l+s+s1}{\PYZsq{}}\PY{p}{]} \PY{o}{=} \PY{n}{dates}\PY{p}{[}\PY{l+s+s1}{\PYZsq{}}\PY{l+s+s1}{date}\PY{l+s+s1}{\PYZsq{}}\PY{p}{]}\PY{o}{.}\PY{n}{dt}\PY{o}{.}\PY{n}{weekday\PYZus{}name}
         
         \PY{c+c1}{\PYZsh{} Create new column `handle` in `dates` DataFrame based on a Date the handle was removed }
         \PY{n}{dates}\PY{p}{[}\PY{l+s+s1}{\PYZsq{}}\PY{l+s+s1}{handle}\PY{l+s+s1}{\PYZsq{}}\PY{p}{]} \PY{o}{=} \PY{o}{.}\PY{o}{.}\PY{o}{.} \PY{o}{\PYZgt{}} \PY{o}{.}\PY{o}{.}\PY{o}{.}
         
         \PY{c+c1}{\PYZsh{} Check the dataset and datatypes}
         \PY{n}{dates}\PY{o}{.}\PY{n}{info}\PY{p}{(}\PY{p}{)}
         
         \PY{c+c1}{\PYZsh{} Create a comparison of how many cholera deaths and attacks there were before and after the handle was removed}
         \PY{n}{dates}\PY{o}{.}\PY{n}{groupby}\PY{p}{(}\PY{p}{[}\PY{l+s+s1}{\PYZsq{}}\PY{l+s+s1}{handle}\PY{l+s+s1}{\PYZsq{}}\PY{p}{]}\PY{p}{)}\PY{o}{.}\PY{n}{sum}\PY{p}{(}\PY{p}{)}
         
         \PY{n}{dates}\PY{o}{.}\PY{n}{head}\PY{p}{(}\PY{p}{)}
\end{Verbatim}


    \begin{Verbatim}[commandchars=\\\{\}]

        ---------------------------------------------------------------------------

        TypeError                                 Traceback (most recent call last)

        <ipython-input-63-fa5c39fecf73> in <module>()
         12 
         13 \# Create new column `handle` in `dates` DataFrame based on a Date the handle was removed 
    ---> 14 dates['handle'] = {\ldots} > {\ldots}
         15 
         16 \# Check the dataset and datatypes
    

        TypeError: unorderable types: ellipsis() > ellipsis()

    \end{Verbatim}

    \subsection{7. The picture worth a thousand
words}\label{the-picture-worth-a-thousand-words}

Removing the handle from the pump prevented any more of the infected
water from being collected. The spring below the pump was later found to
have been contaminated with sewage. This act was later recognized as an
early example of epidemiology, public health medicine and the
application of science (the germ theory of disease) in a real-life
crisis.

A replica of the pump, together with an explanatory and memorial plaque
and without a handle was erected in 1992 near the location of the
original close to the back wall of what today is the John Snow pub. The
site is subtly marked with a pink granite kerbstone in front of a small
wall plaque.

We can learn a lot from John Snow's data. We can take a look at absolute
counts, but this observation could lead us to a wrong conclusion so
let's take a different look on the data using Bokeh.

Thanks to John Snow we have the data in chronological order (i.e. as
time series data), so the best way to see the whole picture is to
visualize it and look at it the way he saw it while writing On the Mode
of Communication of Cholera (1855).

    \begin{Verbatim}[commandchars=\\\{\}]
{\color{incolor}In [{\color{incolor} }]:} \PY{k+kn}{import} \PY{n+nn}{bokeh}
        \PY{k+kn}{from} \PY{n+nn}{bokeh}\PY{n+nn}{.}\PY{n+nn}{plotting} \PY{k}{import} \PY{n}{output\PYZus{}notebook}\PY{p}{,} \PY{n}{figure}\PY{p}{,} \PY{n}{show}
        \PY{n}{output\PYZus{}notebook}\PY{p}{(}\PY{n}{bokeh}\PY{o}{.}\PY{n}{resources}\PY{o}{.}\PY{n}{INLINE}\PY{p}{)}
        
        \PY{c+c1}{\PYZsh{} Set up figure}
        \PY{n}{p} \PY{o}{=} \PY{n}{figure}\PY{p}{(}\PY{n}{plot\PYZus{}width}\PY{o}{=}\PY{l+m+mi}{900}\PY{p}{,} \PY{n}{plot\PYZus{}height}\PY{o}{=}\PY{l+m+mi}{450}\PY{p}{,} \PY{n}{x\PYZus{}axis\PYZus{}type}\PY{o}{=}\PY{l+s+s1}{\PYZsq{}}\PY{l+s+s1}{datetime}\PY{l+s+s1}{\PYZsq{}}\PY{p}{,} \PY{n}{tools}\PY{o}{=}\PY{l+s+s1}{\PYZsq{}}\PY{l+s+s1}{lasso\PYZus{}select, box\PYZus{}zoom, save, reset, wheel\PYZus{}zoom}\PY{l+s+s1}{\PYZsq{}}\PY{p}{,}
                  \PY{n}{toolbar\PYZus{}location}\PY{o}{=}\PY{l+s+s1}{\PYZsq{}}\PY{l+s+s1}{above}\PY{l+s+s1}{\PYZsq{}}\PY{p}{,} \PY{n}{x\PYZus{}axis\PYZus{}label}\PY{o}{=}\PY{l+s+s1}{\PYZsq{}}\PY{l+s+s1}{Date}\PY{l+s+s1}{\PYZsq{}}\PY{p}{,} \PY{n}{y\PYZus{}axis\PYZus{}label}\PY{o}{=}\PY{l+s+s1}{\PYZsq{}}\PY{l+s+s1}{Number of Deaths/Attacks}\PY{l+s+s1}{\PYZsq{}}\PY{p}{,} 
                  \PY{n}{title}\PY{o}{=}\PY{l+s+s1}{\PYZsq{}}\PY{l+s+s1}{Number of Cholera Deaths/Attacks before and after 8th of September 1854 (removing the pump handle)}\PY{l+s+s1}{\PYZsq{}}\PY{p}{)}
        
        \PY{c+c1}{\PYZsh{} Plot on figure}
        \PY{n}{p}\PY{o}{.}\PY{n}{line}\PY{p}{(}\PY{n}{dates}\PY{p}{[}\PY{l+s+s1}{\PYZsq{}}\PY{l+s+s1}{date}\PY{l+s+s1}{\PYZsq{}}\PY{p}{]}\PY{p}{,} \PY{n}{dates}\PY{p}{[}\PY{l+s+s1}{\PYZsq{}}\PY{l+s+s1}{...}\PY{l+s+s1}{\PYZsq{}}\PY{p}{]}\PY{p}{,} \PY{n}{color}\PY{o}{=}\PY{l+s+s1}{\PYZsq{}}\PY{l+s+s1}{red}\PY{l+s+s1}{\PYZsq{}}\PY{p}{,} \PY{n}{alpha}\PY{o}{=}\PY{l+m+mi}{1}\PY{p}{,} \PY{n}{line\PYZus{}width}\PY{o}{=}\PY{l+m+mi}{3}\PY{p}{,} \PY{n}{legend}\PY{o}{=}\PY{l+s+s1}{\PYZsq{}}\PY{l+s+s1}{Cholera Deaths}\PY{l+s+s1}{\PYZsq{}}\PY{p}{)}
        \PY{n}{p}\PY{o}{.}\PY{n}{circle}\PY{p}{(}\PY{n}{dates}\PY{p}{[}\PY{l+s+s1}{\PYZsq{}}\PY{l+s+s1}{date}\PY{l+s+s1}{\PYZsq{}}\PY{p}{]}\PY{p}{,} \PY{n}{dates}\PY{p}{[}\PY{l+s+s1}{\PYZsq{}}\PY{l+s+s1}{...}\PY{l+s+s1}{\PYZsq{}}\PY{p}{]}\PY{p}{,} \PY{n}{color}\PY{o}{=}\PY{l+s+s1}{\PYZsq{}}\PY{l+s+s1}{black}\PY{l+s+s1}{\PYZsq{}}\PY{p}{,} \PY{n}{nonselection\PYZus{}fill\PYZus{}alpha}\PY{o}{=}\PY{l+m+mf}{0.2}\PY{p}{,} \PY{n}{nonselection\PYZus{}fill\PYZus{}color}\PY{o}{=}\PY{l+s+s1}{\PYZsq{}}\PY{l+s+s1}{grey}\PY{l+s+s1}{\PYZsq{}}\PY{p}{)}
        \PY{n}{p}\PY{o}{.}\PY{n}{line}\PY{p}{(}\PY{n}{dates}\PY{p}{[}\PY{l+s+s1}{\PYZsq{}}\PY{l+s+s1}{date}\PY{l+s+s1}{\PYZsq{}}\PY{p}{]}\PY{p}{,} \PY{n}{dates}\PY{p}{[}\PY{l+s+s1}{\PYZsq{}}\PY{l+s+s1}{...}\PY{l+s+s1}{\PYZsq{}}\PY{p}{]}\PY{p}{,} \PY{n}{color}\PY{o}{=}\PY{l+s+s1}{\PYZsq{}}\PY{l+s+s1}{black}\PY{l+s+s1}{\PYZsq{}}\PY{p}{,} \PY{n}{alpha}\PY{o}{=}\PY{l+m+mi}{1}\PY{p}{,} \PY{n}{line\PYZus{}width}\PY{o}{=}\PY{l+m+mi}{2}\PY{p}{,} \PY{n}{legend}\PY{o}{=}\PY{l+s+s1}{\PYZsq{}}\PY{l+s+s1}{Cholera Attacks}\PY{l+s+s1}{\PYZsq{}}\PY{p}{)}
        
        \PY{n}{show}\PY{p}{(}\PY{n}{p}\PY{p}{)}
\end{Verbatim}


    \subsection{8. John Snow's myth \& Did we learn
something?}\label{john-snows-myth-did-we-learn-something}

From the previous interactive visualization, we can clearly see that the
peak of the cholera outbreak happened before removing the handle and it
was already in decline (downside trajectory) before the 8th of September
1854.

This different view on the data is very important because in case that
we compare just absolute numbers this could lead us to wrong conclusion
that removing the handle on Broad Street pump for sure stopped the
outbreak, which is simply not true (it surely did help but did not stop
the outbreak) and John Snow was aware of this (he just did what needed
to be done and never aspired to become a hero).

But people love stories about heroes and other myths (definitely more
than science or data science). According to John Snow's myth, he was the
superhero who in two days defied their equals by hypothesizing that
cholera was a waterborne disease. Despite no one listening to him, he
bravely continued drawing his map, convinced local authorities to remove
the handle of the infected water pump with his findings, and stopped the
outbreak. John Snow saved the lives of many Londoners.

If we take a better look behind this story, we can find also the true
John Snow, who was fighting the disease with limited tools and wanted to
get proof that he was right and "knew something" about cholera. He just
did what he could with limited time and always boiled his water before
drinking.


    % Add a bibliography block to the postdoc
    
    
    
    \end{document}
